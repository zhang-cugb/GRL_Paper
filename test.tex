\documentclass[a4paper]{article}

%% Language and font encodings
\usepackage[english]{babel}
\usepackage[utf8x]{inputenc}
\usepackage[T1]{fontenc}
\usepackage[parfill]{parskip}

%% Sets page size and margins
\usepackage[a4paper,top=1.5cm,bottom=1.5cm,left=2cm,right=2cm,marginparwidth=1.75cm]{geometry}

%% Useful packages
\usepackage{amsmath}
\usepackage{graphicx}
\usepackage[colorinlistoftodos]{todonotes}
\usepackage[colorlinks=true, allcolors=blue]{hyperref}
\newcommand{\matr}[1]{\mathbf{#1}}
\usepackage{subcaption}
\usepackage{subfiles}
\usepackage[round]{natbib}

\title{Partitioning Thresholds in Hybrid Implicit and Explicit Representations of Naturally Fractured Reservoirs}
\author{Daniel Wong Lorng Yon \and Florian Doster \and Sebastian Geiger \and Arjan Kamp}

\begin{document}
	\citep{Priest1993}
	\citep{Lee2001, Lie2015, Moinfar2013}
	
	\citep{Berkowitz2002}. More recently, there has also been 
	interest in NFRs as a potential site for carbon sequestration \citep{March2018}
	
	 \citep{Ezulike2013, Warren1963, Yan2016}. The conversion from discrete fractures to a continuum representation involves an averaging process known as upscaling, which can be performed numerically or analytically \citep{Durlofsky1991,Oda1985,Renard1997,Saevik2013}. I
	 
	 \citep{Oda1985, Saevik2013, Saevik2014}. 
	 
	 For multiscale NFRs, unfortunately, upscaled equivalent permeability fields have been shown to be dependent on the choice of homogenization scale; this will eventually lead to varying flow modelling outcomes \citep{Elfeel2013}. In particular, NFRs tend to contain fractures with various sizes that oftentimes follow a power law distribution \citep{Bonnet2001}. This implies that no Representative Elementary Volume (REV) can be defined, which invalidates the upscaling step necessary in continuum modelling \citep{Berkowitz2002}.
	 
	 \citep{Lee2001, Lie2015, Moinfar2013}
	
	\bibliographystyle{plainnat}
	\bibliography{/Users/danie/Documents/phd-daniel/Bibtex/library}
\end{document}